%===============================================================================
%===============================================================================
%
\clearpage
%
%
\section{Fibres}
%
A Laplacian flow through the biceps geometry is solved and fibres are traced. The number of fibres is <100.
The distance between nodes on each fibre is such that there are 100 elements per centimeter. This leads to a total of 1500-2000 elements per fibre.


\begin{figure}[h!]
  \centering%
  \includegraphics[width=10cm]{../tests/fibers/results/laplace3d_structured_linear.pdf}
  \caption{\lstinline{laplace3d_structured_linear}}
\end{figure} 
%
\lstinputlisting[breaklines,basicstyle=\tiny]{../tests/fibers/results/log_recent_laplace3d_structured_linear.txt}
%
\begin{figure}[h!]
  \centering%
  \includegraphics[width=10cm]{../tests/fibers/results/laplace3d_structured_quadratic.pdf}
  \caption{\lstinline{laplace3d_structured_quadratic}}
\end{figure} 
%
\lstinputlisting[breaklines,basicstyle=\tiny]{../tests/fibers/results/log_recent_laplace3d_structured_quadratic.txt}
%
\begin{figure}[h!]
  \centering%
  \includegraphics[width=10cm]{../tests/fibers/results/laplace3d_unstructured_linear.pdf}
  \caption{\lstinline{laplace3d_unstructured_linear}}
\end{figure} 
%
\lstinputlisting[breaklines,basicstyle=\tiny]{../tests/fibers/results/log_recent_laplace3d_unstructured_linear.txt}
%
\begin{figure}[h!]
  \centering%
  \includegraphics[width=10cm]{../tests/fibers/results/laplace3d_unstructured_quadratic.pdf}
  \caption{\lstinline{laplace3d_unstructured_quadratic}}
\end{figure} 
%
\lstinputlisting[breaklines,basicstyle=\tiny]{../tests/fibers/results/log_recent_laplace3d_unstructured_quadratic.txt}


%===============================================================================
%===============================================================================
