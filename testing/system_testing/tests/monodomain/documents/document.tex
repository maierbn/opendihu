%===============================================================================
%===============================================================================
%
\clearpage
%
\section{Monodomain Equation}
  %
  The monodomain equation is a reaction diffusion equation of the following form:
  %
  \begin{equation}\label{eq:monodomain1d}
    \begin{array}{lll}
      u_t(x,t) &= c\,u''(x,t) + B(\bfy), \qquad x \in [a,b], \quad t \in [0,t_\text{end}],\\[4mm]
      \bfy_t &= G(\bfy),
    \end{array}
  \end{equation}
  %
  where $\bfy$ is a vector of subcellular states and $B(\bfy), G(\bfy)$ are nonlinear operators.
  We have homogeneous Neumann boundary conditions,
  %
  \begin{equation}\label{eq:monodomain1d_bc}
    \begin{array}{lll}
      u_x(x) \cdot \bfn = 0 \quad  x \in \{a,b\}.
    \end{array}
  \end{equation}
  The solution is obtained using the Godunov operator splitting with explicit Euler integration for each equation.
  
  The used parameters are
  \begin{equation*}
    \begin{array}{lll}
      c = \text{Conductivity/(Am*Cm)},\qquad \Omega = [0,50], \qquad t_\text{end}=50.
    \end{array}
  \end{equation*}
  The discretization parameters are
  \begin{table}[h!]
    \begin{center}
      \begin{tabular}{l|l}
        \textbf{Parameter} & \textbf{Value}\\
        \hline
        Number of elements & 500\\
        splitting timestep $dt_\text{3D}$ & $10^{-1}$\\
        timestep of diffusion term $dt_\text{1D}$ & $10^{-5}$\\
        timestep of ODEs $dt_\text{0D}$ & $5\cdot 10^{-5}$
      \end{tabular}
    \end{center}
    \caption{discretization parameters}
    \label{tab:table_monodomain1}
  \end{table}

\subsection{Result summary}
%
\subsubsection{Shorten, input transformed by OpenCOR}
The right hand side operators $B,G$ are chosen as Shorten model. The C code is generated by OpenCOR.
%
\begin{figure}[h!]
  \animation{../tests/monodomain/results/shorten_opencor.mp4}
  \caption{\lstinline{shorten_opencor}}
\end{figure} 
%
\begin{figure}[t]%
  \centering%
  \includegraphics[width=0.9\textwidth,keepaspectratio]{../tests/monodomain/results/shorten_opencor.pdf}%
  \caption{\lstinline{shorten_opencor}}
\end{figure}%
%

\lstinputlisting[breaklines,basicstyle=\tiny]{../tests/monodomain/results/log_recent_shorten_opencor.txt}

% ---------
\subsubsection{Shorten, input transformed by OpenCMISS}
The right hand side operators $B,G$ are chosen as Shorten model. The C code is generated by OpenCMISS.
%
  \begin{figure}[h!]
    \animation{../tests/monodomain/results/shorten_opencmiss.mp4}
    \caption{\lstinline{shorten_opencmiss}}
  \end{figure}
%
\begin{figure}[t]%
  \centering%
  \includegraphics[width=0.9\textwidth,keepaspectratio]{../tests/monodomain/results/shorten_opencmiss.pdf}%
  \caption{\lstinline{shorten_opencmiss}}
\end{figure}%
%

\lstinputlisting[breaklines,basicstyle=\tiny]{../tests/monodomain/results/log_recent_shorten_opencmiss.txt}
% ---------
\subsubsection{Hodgkin-Huxley}
The right hand side operators $B,G$ are chosen as Shorten model. The C code is generated by OpenCOR.
%
  \begin{figure}[h!]
    \animation{../tests/monodomain/results/hodgkin_huxley.mp4}
    \caption{\lstinline{hodgkin_huxley}}
  \end{figure} 
%
\begin{figure}[t]%
  \centering%
  \includegraphics[width=0.9\textwidth,keepaspectratio]{../tests/monodomain/results/hodgkin_huxley.pdf}%
  \caption{\lstinline{hodgkin_huxley}}
\end{figure}%
%

\lstinputlisting[breaklines,basicstyle=\tiny]{../tests/monodomain/results/log_recent_hodgkin_huxley.txt}
% ---------
%
%
%\begin{figure}[h!]
%    \centering 
%    \includegraphics[width=0.9\columnwidth]{examples/example-0001/doc/figures/analytical_solution.eps} 
%    \caption{Results, analytical solution.}
%    \label{example-0001-analytical-solution-fig}
%\end{figure}
%
%===============================================================================
%===============================================================================
