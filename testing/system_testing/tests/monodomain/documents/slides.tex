%===============================================================================
%===============================================================================
%
\clearpage
%
%
%
%%======================================================================
%
\begin{frame}
  \section{Monodomain Equation}
  %
  The monodomain equation is a reaction diffusion equation of the following form:
  %
  \begin{equation}\label{eq:monodomain1d}
    \begin{array}{lll}
      u_t(x,t) &= c\,u''(x,t) + B(\bfy), \qquad x \in [a,b], \quad t \in [0,t_\text{end}],\\[4mm]
      \bfy_t &= G(\bfy),
    \end{array}
  \end{equation}
  %
  where $\bfy$ is a vector of subcellular states and $B(\bfy), G(\bfy)$ are nonlinear operators.
  We have homogeneous Neumann boundary conditions,
  %
  \begin{equation}\label{eq:monodomain1d_bc}
    \begin{array}{lll}
      u_x(x) \cdot \bfn = 0 \quad  x \in \{a,b\}.
    \end{array}
  \end{equation}
  The solution is obtained using the Godunov operator splitting with explicit Euler integration for each equation.
\end{frame}
%
%======================================================================
%
\begin{frame}
  %
  The used parameters are
  \begin{equation*}
    \begin{array}{lll}
      c = \text{Conductivity/(Am*Cm)},\qquad \Omega = [0,50], \qquad t_\text{end}=50.
    \end{array}
  \end{equation*}
  The discretization parameters are
  \begin{table}[h!]
    \begin{center}
      \begin{tabular}{l|l}
        \textbf{Parameter} & \textbf{Value}\\
        \hline
        Number of elements & 500\\
        splitting timestep $dt_\text{3D}$         & $10^{-1}$\\
        timestep of diffusion term $dt_\text{1D}$ & $10^{-5}$\\
        timestep of ODEs $dt_\text{0D}$           & $5\cdot 10^{-5}$
      \end{tabular}
    \end{center}
    \caption{discretization parameters}
    \label{tab:table_monodomain1}
  \end{table}
\end{frame}
%
%======================================================================
%
% OLD (using multimedia package):
% Path to mp4 files must begin with tests/
% The final pdf only shows the movies if it is located under opendihu/testing/functional_testing/
%
% NEW (using movie15 package): 
% The paths must be begin with ../../tests, the file is included in the final pdfs
%
% Note that there is a warning that says "Package `movie15' is obsolete and superseded by `media9'", but media9 is incompatible with most viewers on linux.
%
\begin{frame}
  \frametitle{Shorten, input transformed by OpenCOR}
  The right hand side operators $B,G$ are chosen as Shorten model. The C code is generated by OpenCOR.
  \vspace*{-0.2cm}
  \begin{figure}[h!]
    \begin{subfigure}[t]{0.5\textwidth}%
    \centering
    \animation{../../tests/monodomain/results/shorten_opencor.mp4}
    \end{subfigure}
    \begin{subfigure}[t]{0.48\textwidth}%
      \centering%
      \includegraphics[height=0.7\textheight,width=\textwidth,keepaspectratio]{../../tests/monodomain/results/shorten_opencor.pdf}%
    \end{subfigure}%
    \caption{\lstinline{shorten_opencor}}
  \end{figure} 
  \showlog{../../tests/monodomain/results/log_recent_shorten_opencor.txt}
\end{frame}
%
%======================================================================
%
\begin{frame}
  \frametitle{Shorten, input transformed by OpenCMISS}
  The right hand side operators $B,G$ are chosen as Shorten model. The C code is generated by OpenCMISS.
  \vspace*{-0.2cm}
  \begin{figure}[h!]
    \begin{subfigure}[t]{0.5\textwidth}%
    \centering
    \animation{../../tests/monodomain/results/shorten_opencmiss.mp4}
    \end{subfigure}
    \begin{subfigure}[t]{0.48\textwidth}%
      \centering%
      \includegraphics[height=0.7\textheight,width=\textwidth,keepaspectratio]{../../tests/monodomain/results/shorten_opencmiss.pdf}%
    \end{subfigure}%
    \caption{\lstinline{shorten_opencmiss}}
  \end{figure} 
  \showlog{../../tests/monodomain/results/log_recent_shorten_opencmiss.txt}
\end{frame}
%
%======================================================================
%
\begin{frame}
  \frametitle{Hodgkin-Huxley}
  The right hand side operators $B,G$ are chosen as Hodgkin Huxley model. The C code is generated by OpenCOR.
  \vspace*{-0.2cm}
  \begin{figure}[h!]
    \begin{subfigure}[t]{0.5\textwidth}%
    \centering
    \animation{../../tests/monodomain/results/hodgkin_huxley.mp4}
    \end{subfigure}
    \begin{subfigure}[t]{0.48\textwidth}%
      \centering%
      \includegraphics[height=0.7\textheight,width=\textwidth,keepaspectratio]{../../tests/monodomain/results/hodgkin_huxley.pdf}%
    \end{subfigure}%
    \caption{\lstinline{hodgkin_huxley}}
  \end{figure} 
  \showlog{../../tests/monodomain/results/log_recent_hodgkin_huxley.txt}
\end{frame}
%
%===============================================================================
%===============================================================================
